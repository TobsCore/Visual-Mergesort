\chapter{Weiterentwicklung}\label{chap:weiterentwicklung}
Falls man das Programm manuell kompilieren möchte, oder man die Funktionen weiterentwickeln möchte, haben wir hier eine kleine Anleitung: Welche Programme man benötigt, wo welche Dateien liegen, wie man das Programm manuell kompiliert und wie man schlussendlich eine ausführbare Datei bekommt.

\subsection{Voraussetzungen}

Wenn man das Programm selber kompilieren und weiter-entwickeln möchte, benötigt man die folgenden Dinge:

\begin{enumerate}
\item SBT (Scala Build Tool)
\item git
\item JVM und JDK
\end{enumerate}

Wir gehen davon aus, dass sowohl JVM als auch JDK installiert sind. Alternativ kann man auf der Oracle--Seite\cite{OracleInstallJDK} nachlesen, wie man diese für sein Betriebssystem installiert.

Das Scala Built Tool kann man auf der Projektseite\cite{InstallSBT} herunter laden, und im Anschluss kann man dieses installieren. Über dieses Build Tool wird dann später Scala in der entsprechenden Version geladen, sodass man Scala selber nicht installieren muss. Möchte man Scala dennoch installieren, um beispielsweise die \texttt{REPL} \cite{GettingStartedWithTheScalaREPL} zu nutzen, so kann man Scala über die Webseite\cite{ScalaLang} herunter laden.

Zuletzt benötigt man noch git\cite{GitManual}, um das Projekt über Github beziehen zu können.

\subsection{Programm starten}

Im Nachfolgenden werden die Befehle aus dem Terminal ausgeführt. Diese Schritt-für-Schritt--Anleitung wurde auf Linux und OSX getestet, sollte jedoch auf Windows ähnlich funktionieren.

Zuerst muss das Projekt von Github geladen werden. Über git kann man den folgenden Befehl ausführen, um das Projekt von Github zu \texttt{clonen}.

\begin{verbatim}
git clone https://github.com/TobsCore/Visual-Mergesort.git
\end{verbatim}

Es ist ein neues Verzeichnis mit den Projektdaten erstellt worden. Wenn man in dieses Verzeichnis navigiert, befinden sich dort zwei Ordner und eine \texttt{README.md} Datei.

\begin{verbatim}
cd Visual-Mergesort/
ls
\end{verbatim}

In den Ordnern befinden sich die folgenden Daten:

\begin{description}
\item[Code/] In diesem Ordner ist das eigentliche Programm enthalten, sowie die sbt--Kon\-fi\-gu\-ra\-tion und die Intellij Dateien.
\item[Documentation/] Hier ist die Dokumentation über unsere Projektarbeit enthalten, die wir geschrieben habe. Die Dokumentation ist in \LaTeX~ geschrieben und kann über \texttt{xelatex} kompiliert werden. Mehr dazu in \ref{sec:latex}.
\end{description}

Wenn man nun in den \texttt{Code/} Ordner wechselt, befinden sich darin wiederrum 4 Ordner (3 sichtbar, 1 nicht sichtbar):

\begin{description}
\item[project/] Hier sind die sbt Projektdateien enthalten.
\item[src/] Dies ist der wichtigste Unterordner. Hier ist der Programmcode enthalten, da es sich um den \texttt{Source}--Ordner handelt.
\item[target/] Hier landen kompilierte Dateien und Caches. Die ausführbare \texttt{Visual-Mergesort"".jar}--Datei wird beispielsweise auch in diesen Ordner geschrieben.
\item[.idea/] Dies ist der Intellij Projektordner.
\end{description}

Zuletzt möchten wir noch die Ordnerstruktur des \texttt{src/} Ordners erklären. In dem Unterordner \texttt{main/} befinden sich noch die Ordner \texttt{resources/} und \texttt{scala/}. In \texttt{resources/}, sind Bilder, die CSS Dateien und die \texttt{FXML} Dateien enthalten. Der ganze Programmcode ist in dem Ordner \texttt{scala/}.

Um das Programm nun kompilieren zu können, gehen wir zurück in den Ordner \texttt{Visual""-Mergesort""/Code/}. In diesem Ordner sollte sich eine Datei mit dem Namen \texttt{build.sbt} befinden. Wenn man nun den Befehl

\begin{verbatim}
sbt compile
\end{verbatim}

ausführt, so wird der Programmcode kompiliert und kann im Anschluss über

\begin{verbatim}
sbt run
\end{verbatim}

ausgeführt werden. Nun wird die Anwendung gestartet.

\subsubsection{Ausführbare Datei erstellen}
Da man nicht immer den Programmcode beziehen und kompilieren möchte, ist es sinnvoll, wenn man sich eine ausführbare Anwendung erstellen lässt. Dies ist, wie bereits erwähnt, eine Datei mit der Endung \texttt{.jar}. Wir benutzen ein sbt--Plugin, das sich um die Erstellung dieser Datei kümmert. Man kann die Erstellung über

\begin{verbatim}
sbt assembly
\end{verbatim}

anstoßen. Die generierte Datei liegt dann in dem Ordner \texttt{target/scala-2.11/} und heißt \texttt{Visual""-Mergesort"".jar}. Mit

\begin{verbatim}
java -jar Visual-Mergesort.jar
\end{verbatim}

kann man die Anwendung starten.

\subsection{Dokumentation}\label{sec:latex}

Die Dokumentation ist in \LaTeX ~geschrieben und kann über \texttt{xelatex} erstellt werden. Das Programm kann auf Linux, Windows und OSX installiert werden. Wir haben zusätzlich ein \texttt{Makefile} angelegt, das das Kompilieren und anschließende Aufräumen für uns übernimmt. Zusätzlich kann man hierüber auch die \texttt{Bibtex} Datei in das Dokument einbauen. \texttt{BibTex} wird genutzt, um die Literaturquellen zu verwalten. Die Datei \texttt{Documentation/README.md}\footnote{Auch hier zu finden: \url{https://github.com/TobsCore/Visual-Mergesort/blob/master/Documentation/README.md}} beschreibt, welche Befehle man über Make\footnote{Es kann sein, dass man das Make--Progamm installieren muss, um das Makefile nutzen zu können. Dies war bei uns zumindest auf Windows--Systemen der Fall} ausführen kann und was diese Befehle bewirken.
