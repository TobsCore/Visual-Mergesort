\chapter{Einleitung}\label{chap:intro}
\section{Aufgabenbeschreibung}
Die Aufgabe in unserer Projektarbeit besteht darin, den Mergesort-Algorithmus in der Programmiersprache Scala zu implementieren sowie dessen Ablauf visuell darzustellen.
\section{Motivation}
Schon vor der Themenauswahl für unsere Projektarbeit waren wir von der Programmiersprache Scala begeistert. Um diese Programmiersprache zu erlernen, suchten wir nach einem Thema, welches sinnvollerweise in Scala implementiert werden könnte. Mit den Neuerungen von ScalaFX kam uns die Aufgabe zur Visualisierung eines parallelisierbaren Sortieralgorithmus sehr entgegen und verleitete uns schließlich zu unserer Entscheidung für dieses Thema.

Da der Mergesort ein sehr performanter und oft genutzter Sortieralgorithmus für eine große Menge an Elementen ist, fanden wir es interessant, etwas tiefer in die Materie einzutauchen und den Algorithmus in seinen Einzelheiten genau zu untersuchen, sowie diese visuell als Animationen darzustellen.\\
Desweiteren war es uns von großer Bedeutung, während der Projektarbeit an praktischer Programmiererfahrung dazuzugewinnen und auch theoretisch etwas Neues zu lernen. Deshalb haben wir uns auch dazu entschieden, die Aufgabe nicht in Java, sondern in einer anderen, modernen Programmiersprache zu implementieren, welche wir uns neu aneignen mussten. Darüber hinaus ist die Arbeitsweise mit \texttt{JavaFX und ScalaFX} sehr interessant, wenn es um das erstellen von Animationen und Transitionen geht. Zusätzlich war es uns wichtig, ein schönes User-Interface zu implementieren, um die dadurch erlangten Kenntnisse bei Gelegenheit in der Zukunft anwenden zu können.\\
Wir wissen, dass der Mergesort aufgrund seiner rekursiven Implementierung für viele nicht auf Anhieb zu verstehen ist. Deshalb besteht unser Ziel bei dieser Projektarbeit darin, den Benutzer beim Verstehen des Mergesort-Algorithmus so weit wie möglich zu unterstützen und ihm mit zahlreichen Features den Eintritt in die Welt der rekursiven Programmierung zu erleichtern. Zusätzlich zeigen wir auf, wie ein parallelisierbares Problem, in diesem Fall das Sortieren einer Liste, mit Hilfe von mehreren Threads effizienter gelöst werden kann.

\section{Methodische Vorgehensweise}
Da wir den Visual Mergesort in Scala implementiert haben, wird in dieser Dokumentation zuerst auf die Sprache an sich eingegangen. Im zweiten Kapitel werden also zunächst die grundlegenden Konzepte von Scala erläutert. Um das Einfinden in Scala zu erleichtern, werden wir zudem die Besonderheiten der Syntax genau erklären und gegebenenfalls einem vergleichbaren Beispiel der bekannten Programmiersprache Java gegenüberstellen. Im dritten Kapitel schaffen wir dem Leser in den Bereichen Rekursion und Mergesort-Algorithmus eine theoretische Grundlage, die benötigt wird, um den internen Systemablauf während der Animation nachvollziehen zu können. Im vierten Kapitel bekommt der Leser einen Einblick in den Aufbau unseres Programms. Darüber hinaus werden bestimmte Klassen und Codestücke, welche für die Implementierung eine zentrale Rolle spielen, ausführlich erklärt. Im fünften Kapitel befindet sich eine Bedienungsanleitung, welche sich ausschließlich mit der Bereitstellung und Benutzung der Applikation befasst. Zudem werden dort die nötigen Schritte erklärt, um das Programm selber zu kompilieren. Zuallerletzt werden wir unsere Vorgehensweise bei der Implementierung reflektieren und diese sowie unsere Ergebnisse einer kritischen Betrachtung unterziehen.
