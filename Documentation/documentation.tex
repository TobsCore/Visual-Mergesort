% report.tex
% main file for the project report.

\documentclass[a4paper,titlepage,12pt]{scrreprt}

% utf-8
\usepackage{polyglossia}
\setdefaultlanguage[babelshorthands]{ngerman}
\usepackage{fontspec}
\usepackage{pdfpages}

%Needed to wrap text around images, used for personas
\usepackage{wrapfig}

%Used to avoid page breaks with each chapter
\usepackage{etoolbox}
\makeatletter
\patchcmd{\chapter}{\if@openright\cleardoublepage\else\clearpage\fi}{}{}{}
\makeatother

% german names
\usepackage{ngerman}

% colored links
\usepackage{color}
\usepackage[colorlinks=true, citecolor=red] {hyperref}
\definecolor{grey}{rgb}{0.2,0.2,0.2}
\definecolor{orange}{rgb}{1,0.3,0}
\definecolor{turqoise}{rgb}{0,0.7,0.5}

% code listings
\usepackage{listings}
% "define" Scala
\lstdefinelanguage{scala}{
  morekeywords={abstract,case,catch,class,def,%
    do,else,extends,false,final,finally,%
    for,if,implicit,import,match,mixin,%
    new,null,object,override,package,%
    private,protected,requires,return,sealed,%
    super,this,throw,trait,true,try,%
    type,val,var,while,with,yield},
  otherkeywords={=>,<-,<\%,<:,>:,\#,@},
  sensitive=true,
  morecomment=[l]{//},
  morecomment=[n]{/*}{*/},
  morestring=[b]",
  morestring=[b]',
  morestring=[b]"""
}

% Style languages
\lstset{%
	language=Java,
	basicstyle={\ttfamily \small},
	breaklines=true,
	commentstyle=\color{grey},
	keywordstyle=\color{orange},
	numbers=left,
	showspaces=false,
	stringstyle=\color{turqoise},
	xleftmargin=20pt
}

\lstset{%
	language=scala,
	basicstyle={\ttfamily \small},
	breaklines=true,
	commentstyle=\color{grey},
	keywordstyle=\color{orange},
	numbers=left,
	showspaces=false,
	stringstyle=\color{turqoise},
	xleftmargin=20pt
}


% graphics
\usepackage{graphicx}
\graphicspath{{images/}}

% BibTex lib - used for citation
\usepackage{cite}
% fancy headers and footers
\usepackage{fancyhdr}
\pagestyle{fancy}
% clear style
\fancyhead{}
\fancyfoot{}
% new style
\renewcommand{\chaptermark}[1]{%
	\markboth{\thechapter.\ #1}{}
}
\renewcommand{\sectionmark}[1]{%
	\markright{\thesection.\ #1}{}
}
\renewcommand{\headrulewidth}{0.5pt}
\renewcommand{\footrulewidth}{0.5pt}
\fancyhead[LE,RO]{\rightmark}
\fancyhead[LO,RE]{\leftmark}
\fancyfoot[LE,RO]{\thepage}
\fancyfoot[LO,RE]{Projektarbeit T. Kerst \& P. König --- Sommersemester 2016}
\fancypagestyle{plain}{%
	\fancyhf{}
	\renewcommand{\headrulewidth}{0pt}
	\renewcommand{\footrulewidth}{0.5pt}
	\fancyfoot[LE,RO]{\thepage}
	\fancyfoot[LO,RE]{Projektarbeit T. Kerst \& P. König --- Sommersemester 2016}
}

% no indented paragraphs
\usepackage{parskip}

\setkomafont{disposition}{\normalfont\bfseries}

% for verbatiminput
\usepackage{verbatim}

% not yet used
%\input{src/cmd}

\begin{document}

\titlehead{
	\includegraphics[width=0.9\linewidth]{hska_logo}
}

\title{Ausarbeitung Projektarbeit}
\subtitle{Multithreadingfähiger Mergesort-Algorithmus in Scala mit Visualisierung über eine grafische Oberfläche}
\author{%
	Tobias Kerst \\
	Patrick König
}
\date{Sommersemester 2016}
\publishers{
    \textbf{Dozent:} Prof. Dr. Heiko Körner
}
\maketitle

\clearpage

\begingroup
\hypersetup{linkcolor=black}
\tableofcontents
\endgroup

\clearpage

\chapter{Einleitung}\label{chap:intro}
\section{Motivation}

\section{Scala}
Wir haben uns entschieden, das Programm in Scala zu schreiben, einer Sprache, die bereits schon im Jahr 2001 von Martin Odersky entwickelt wurde, jedoch erst seit kurzem einen großen Bekanntheitsgrad erlangt hat. Grund dafür ist der Hype um die sogenannte \texttt{funktionale Programmierung} und das Bedürfnis, Anwendungen nebenläufig zu entwickeln. 
Wir möchten nun zu Beginn die Gründe nennen, warum Scala sich als moderne Programmiersprache anbietet, die sogar von Javas Hauptentwickler James Gosling als bevorzugte Java-Alternative betitelt wurde \footnote{\url{http://www.adam-bien.com/roller/abien/entry/java_net_javaone_which_programming}}.

\subsection{Laufzeit-Umgebung}
Scala ist wie Java eine Programmiersprache, die zu Bytecode kompiliert wird. Dieser Bytecode wird dann von der \textit{Java Virtual Machine} (\texttt{JVM}) benutzt um daraus Maschinencode zu erstellen. Die JVM ist mittlerweile auf sehr vielen Rechnern installiert und sogar das Android Betriebssystem setzt auf eine Variante der \texttt{JVM} (\texttt{Dalvik}).

Die JVM hat den großen Vorteil, dass sie mittlerweile seit 20 Jahren aktiv entwickelt wird und die gesamte Java-Umgebung besonders im Enterprise Bereich eingesetzt wird. Das Resultat ist ein sehr stabiles und vor allem perfomantes System, das aus dem heutigen IT-Markt nicht mehr wegzudenken ist. 

\subsection{Bibliotheken}
Scala wird also nicht zu irgendeinem Bytecode kompiliert, sondern zu Java-Bytecode, um genau zu sein. Dies hat den großen Vorteil, dass man neben der JVM-Unterstützung auch auf etablierte Java Bibliotheken zugreifen kann. Beispielsweise ist der Einsatz von Google GSON\footnote{Link zu der Projektseite von Google GSON: \url{https://github.com/google/gson}}, das das Serialisieren von Objekten in JSON und zurück ermöglicht, über Scala so möglich, wie über Java.

So ist es dann auch kaum überraschend, dass der Einsatz von den üblichen Java Bibliotheken in Scala möglich ist. Ein Beispiel ist \texttt{JavaFX}, welches seit Java 8 Teil des \textit{Java Development Kit} (\textit{JDK}) ist.

\subsection{JavaFX und ScalaFX}
\texttt{JavaFX 8} ist der offzielle Nachfolger von \texttt{Swing}. Zu Beginn sollte es eine Scriptssprache werden, jedoch wurde dieser Fokus mit Version 2 aufgegeben und JavaFX wurde zu der GUI Bibliothek, wie man sie heute nutzt. In der aktuellen Version 8 (welcher der direkte Nachfolger von Version 2 ist und wegen der Einbindung in das JDK 8 diesen Versionssprung vollzogen hat), wurde die Bibliothek um wichtige Komponenten erweitert und bietet die folgendenden Vorteile: 

\begin{description}
\item[Scene Graph] JavaFX ist besonders leicht zu entwickeln, da es auf den \texttt{Scene Graph} setzt. Der Scene Graph ist eine Baumstruktur, bei der die Elemente hierachisch angeordnet werden. Elemente im Scene Graph sind vom Type \texttt{Node} ~\cite{ProJavaFX8}. Wenn man also beispielsweise einen Button auf der Bildfläche platzieren möchte, dann hat man ein Button Objekt, welches von \texttt{Node} erbt

\begin{figure}[!htb]
    \centering
      \includegraphics[width=0.75\linewidth]{scene-graph}
    \caption{Darstellung der Baumstruktur im Scene Graph}
    \label{fig:scene-graph}
\end{figure}

Ein Vorteil ist, dass man so Objekte gruppieren kann und dadurch Operationen auf der Gruppe ausführen kann. So wird bei einer Verschiebe-Operation (auch \texttt{Translation} genannt) jedes Element in der Gruppe verschoben, was den Code lesbarer und wartbarer macht.\
Außerdem ist somit eine zum Eltern Element relative Positionierung möglich.

%TODO: Section FXML einbinden
\item[Klares MVC] Bei JavaFX wird das Konzept für das Design in einer separaten Datei, einer \texttt{FXML} Datei geführt. So muss man das Layout nicht im Code generieren. Im Abschnitt zu FXML~\ref{sec:fxml} wird dies ausführlicher beschrieben.

\end{description}

\subsection{FXML}\label{sec:fxml}

\section{Ziele \& Fragen}
\section{Methodische Vorgehensweise}
\clearpage

\chapter{Theoretische Grundlgen}\label{chap:theoretische-grundlagen}
\section{Rekursion}
Um den Mergesort-Algorithmus in seiner eleganten Form zu verstehen, ist es unausweichlich, sich zuerst mit dem
Konzept der Rekursion zu befassen. Generell bedeutet Rekursion, dass ein Regelwerk erneut auf die Dinge angewendet werden kann,
welche mit diesem Regelwerk erzeugt wurden, was potenziell zu einer endlosen Produktionsschleife, wie beispielsweise bei einer Rückköpplung
führen kann. In der Informatik dient Rekursion dazu, sonst sehr komplexe Sachverhalte elegant zu formulieren. Vereinfachterweise ist eine rekursive Methode
eine Methode, die sich selbst aufruft und somit der Gegenspieler zu einer iterativen Methode. Hierbei spielt die korrekt formulierte Abbruchbedingung der Methode eine zentrale Rolle, da man sonst Gefahr läuft,
eine Endlosschleife zu produzieren. Prinzipiell lässt sich sagen, dass iterative und rekursive Programmierung gleich mächtig sind, da sich jedes rekursiv lösbare Problem
unter mehr oder weniger Umständen auch iterativ ausformulieren lässt und umgekehrt.

Ein Beispiel hierfür wäre die einfache Berechnung der Fakultät einer Zahl in Java. Die erste Lösung erfolgt rekursiv:

\begin{lstlisting}[language=Java]
  public static berechne_fakultaet_rekursiv(int n){

    if(n <= 1){
    return 1;
    }
    else{
    return ( n * berechne_fakultaet_rekursiv(n-1));
    }

  }
\end{lstlisting}

Der rekursive Methodenaufruf befindet sich direkt hinter dem zweiten \texttt{return} Ausdruck. Hier wird die gleich Methode immer wieder
mit einer um eins dekrementierten Zahl aufgerufen, bis die vordefinierte Abbruchbedingung $n <= 1$ eintritt.

Alternativ lässt sich die gleiche Funktion auch iterativ implementieren, wobei die Methode nur ein einziges Mal aufgerufen wird, und das Problem linear gelöst wird:

\begin{lstlisting}[language=Java]
  public static berechne_fakultaet_iterariv(int n){

  int fakultaet = 1;
  int faktor = 2;
  while (faktor <= n){

      fakultaet = fakultaet * faktor;
      faktor ++;

  }
  return fakultaet;

  }
\end{lstlisting}

Wie man sieht wurde der rekursive Methodenaufruf in diesem Fall durch eine while-Schleife ersetzt.


\section{Merge--Sort}
Der erstmals 1945 durch John von Neumann vorgestellte Mergesort ist ein Sortieralgotithmus, der nach dem Paradigma \textit{divide--and--conquer} arbeitet. Bei diesem Prinzip wird das eigentliche, große Problem so lange rekursiv in kleinere Probleme unterteilt, bis diese lösbar sind. Im Anschluss wird aus allen Teillösungen die Endlösung rekonstruiert. Die genaue Funktionsweise des Mergesort erfolgt in zwei Schritten: Im ersten Schritt wird die ursprüngliche Liste in zwei Hälften zerlegt, die jeweils wieder in einer Liste gespeichert werden. Dieser Schritt wird, zusammen mit dem nachfolgenden Schritt, so lange rekursiv fortgesetzt, bis sich nur noch ein Element in jeder Liste befindet. Im zweiten Schritt werden die Hälften sortiert und zu einer Menge zusammengefügt, bis sich irgendwann wieder die Gesamtmenge mit allen enthaltenen Ursprungselementen ergibt. Hierbei werden immer die ersten Elemente der beiden Hälften verglichen, wobei das jeweils kleinere in die zusammengefügte Menge wandert.


\subsection{Laufzeit}

Da sich die Größe der Liste bei jedem Merge verdoppelt, werden $log(n)$ ($n$ = Anzahl der zu sortierenden Elemente) Mergeschritte benötigt, um das Ergebnis vollständig zusammenzusetzen. Bei beispielsweise 8 Elementen ergeben sich insgesamt $log_2 (8) = 3$ Schritte.

Jeder Mergeschritt benötigt wiederum $n$ Schritte, um die Elemente der beiden Listen zu sortieren, da hierzu jedes einzelne Elemente betrachtet und eingeordnet werden muss

Die gesamte Laufzeit beträgt also O($n log (n) $). Im Vergleich zu anderen Sortieralgorithmen wie Bubblesort (Worst-Case-Laufzeit: O($n^2$)) und Quicksort (Worst-Case-Laufzeit: O($n^2$)) ist der Mergesort bei größeren Datenmengen sehr effizient, da dessen Laufzeiten im Best-Case- und im Worst-Case-Szenario kaum Unterschiede aufweisen.

\subsection{Allgemeine Implementierung in Scala}
\subsubsection{Sortieren}

\begin{lstlisting}[language=Scala]
def sort(list: List[SortElement]) {
    if (list.size > 1 ){

      val firstListLength = (list.size / 2.0).ceil.toInt
      val splitList = list.splitAt(firstListLength)
      val left = splitList._1
      val right = splitList._2
      sort(left)
      sort(right)
      merge(list, left, right)
    }
  }
\end{lstlisting}

Die Methode \texttt{sort} nimmt eine Liste von Elementen entgegen und teilt diese in der Mitte in zwei weitere Listen auf. Durch \texttt{list.splitAt} wird ein Tupel erzeugt, welches in der Value splitList gespeichert wird. Mit Hilfe eines Unterstrichs kann entweder auf das erste Feld oder auf das zweite Feld des Tupels zugegriffen werden. Nachdem die ursprüngliche Liste gesplittet wurde, wird auf dem linken Teil rekursiv die Methode \texttt{sort} aufgerufen. Dies sorgt zunächst dafür, dass die linke Liste so lange in zwei Teile aufgeteilt wird, bis sich nur noch ein einziges Element in dieser Liste befindet und die Abbruchbedingung der Rekursion greift. Zu diesem Zeitpunkt befindet sich im rechten Teil der Liste auch nur ein Element und es kommt zur ersten Zusammensetzung zweier sortierter Listen. Diese Methodik wird nun rekursiv für alle Teile der Liste durchgeführt, bis die zwei zu Anfang aufgeteilten Listen sortiert sind und es zum letzten Merge kommt.

\subsubsection{Mergen}
\begin{lstlisting}[language=Scala]
def merge(resultList:List[SortElement], leftList:List[SortElement], rightList:List[SortElement]) {

  val leftSize: Int = leftList.size
  val rightSize: Int = rightList.size
  val totalSize: Int  = leftSize + rightSize
  var i = 0
  var j = 0

  for (k <- 0 until totalSize) {

    if(i < leftSize && j < rightSize){

      if (leftList(i) < rightList(j)){
        resultList = resultList.updated(k, leftList(i))
        i = i + 1
      } else {
        resultList = resultList.updated(k, rightList(j))
        j = j + 1
      }
    } else if (i >= leftSize && j < rightSize){
      resultList = resultList.updated(k, rightList(j))
      j = j + 1
    } else {
      resultList = resultList.updated(k, leftList(i))
      i = i + 1
    }
  }
}
\end{lstlisting}

Diese Methode nimmt zwei vorsortierte Listen an und setzt alle Elemente dieser Teillisten zu einer einzigen, sortierten Liste zusammen. Hierzu wird für jeden Index $k$ der Ergebnisliste bestimmt, welches Element aus den beiden Listen an dieser Stelle einsortiert wird. Vorausgesetzt, die Ergebnisliste enthält noch nicht alle Elemente mindestens einer Teilliste, werden bei jedem Schritt die jeweils kleinsten Elemente der beiden Teillisten miteinander verglichen, die sich noch nicht in der Ergebnisliste befinden. Das Kleinere wird an den aktuellen Index $k$ der Ergebnisliste gesetzt und der Zeiger $i$ bzw. $j$ der Teilliste wird inkrementiert, um auf das nächstgrößere Element zu zeigen. An dieser Stelle macht man es sich zunutze, dass die beiden Teillisten vorsortiert sind, da das nächstgrößere Element mit dem Hochzählen des Indexes $i$ bzw. $j$ der Teillisten automatisch ausgewählt wird. In diesem Codebeispiel existiert also eine einzige Liste, die fortlaufend aktualisiert wird, bis sie das Ergebnis beinhaltet. Zur Visualisierung des Algorithmus in unserer Applikation werden wir später für jedes Zwischenergebnis, welches gesplittet oder zusammengesetzt wurde, eine eigene Gruppe anlegen.

\clearpage

\chapter{Eigenleistung}\label{chap:eigenleistung}

\section{Das Objekt VisualMergesort}
Das Objekt VisualMergesort ist ein Singleton-Objekt der anonymen Klasse VisualMergesort, welches garantiert einzigartig ist. Das bedeutet, dass keine weiteren Objekte gleicher Art erzeugt werden können. „VisualMergesort“ erbt von JFXApp und definiert somit den Startpunkt unserer Applikation.

\begin{lstlisting}[language=Scala]
import java.io.IOException

import scalafx.application.JFXApp
import scalafx.application.JFXApp.PrimaryStage
import scalafx.scene.Scene
import scalafx.Includes._
import scalafxml.core.{FXMLView, NoDependencyResolver}
object VisualMergesort extends JFXApp {

  private val layoutFile: String = "/VisualMergesort.fxml"
  val resource = getClass.getResource(layoutFile)

  if (resource == null) {
    throw new IOException(s"Cannot load resource: $layoutFile")
  }

  val root = FXMLView(resource, NoDependencyResolver)

  stage = new PrimaryStage() {
    title = "Visual Mergesort"
    scene = new Scene(root)
  }

}
\end{lstlisting}

In diesem Objekt wird das in FXML vordefinierte Layout unserer GUI in die Stage der Applikation geladen.

\section{Die Klasse SortElement}
Die Klasse SortElement ist für die Objekte verantwortlich, die durch den Algorithmus auf der Zeichenfläche sortiert werden sollen. Die zu sortierenden Objekte sind zusammengesetzte Gruppen und bestehen aus einem Rechteck sowie einem darunter liegenden Text, welcher den Wert des Objektes angibt. Die Höhe des Rechtecks verhält sich proportional zu diesem Wert.
Der Text benötigt, um mittig unter dem Rechteck zu stehen, noch einen offset, falls der Wert einstellig ist.

%TODO: Caption
\begin{lstlisting}[language=Scala]
class SortElement(val number: Int, var _xPos: Double, var _yPos: Double) extends Group with Ordered[SortElement]  {
  require(number >= 1 && number <= 99 , "the number must be between 1 and 99 (inclusive)")

  var text = new Text(number.toString)
  text.style = "-fx-font-size: 10px; -fx-background: #f00"

  val offset = if (number < 10) { SortElement.smallNumberOffset } else { 0 }
  text.translateX() = xPos + offset
  text.translateY() = _yPos + number + SortElement.width

  var rectangle = new Rectangle(new javafx.scene.shape.Rectangle(_xPos, _yPos, SortElement.width, number))
  rectangle.setFill(Color.DARKBLUE)
  this.getChildren.addAll(rectangle,text)
\end{lstlisting}

Um die von Scala vordefinierten Getter und Setter der Variablen \texttt{xPos} und \texttt{yPos} zu überschreiben, werden diese in \texttt{\_xPos} und
\texttt{\_yPos} umbenannt. Danach werden zunächst die Getter definiert:

\begin{lstlisting}[language=Scala,caption=Definiert die Setter Methoden]
def xPos = _xPos
def yPos = _yPos
\end{lstlisting}

Dieser Code definiert zwei simple Methoden \texttt{xPos} und \texttt{yPos}, welche die Variablen \texttt{\_xPos} und \texttt{\_yPos} zurückgeben. Da bei Scala der letzte Ausdruck einer Methode auch gleichzeitig der Rückgabewert ist und geschweifte Klammern nicht benötigt werden, falls die Methode nur aus einem Ausdruck besteht, sind diese und das Return-Statement nicht vorhanden.

Als nächstes werden die Setter neu definiert:

%TODO: Caption
\begin{lstlisting}[language=Scala]
def xPos_= (x: Double) {
  _xPos = x
  text.translateX = _xPos + offset
  rectangle.x = x
}

def yPos_= (y: Double) {
  _yPos = y
 text.translateY = _yPos + number + SortElement.width
  rectangle.y = y
}
\end{lstlisting}

Der Name dieser Methoden ist \texttt{xPos\_} und \texttt{yPos\_}. Der Unterstrich dient in Scala als spezielles Zeichen und ist in diesem Fall als Platzhalter zu verstehen, der beim Aufrufen der Methoden durch ein Leerzeichen ersetzt werden kann. Somit können diese Methoden im folgenden Code sowohl mit \texttt{xPos =}  bzw. \texttt{yPos =} als auch mit \texttt{xPos\_=} bzw. \texttt{yPos\_=} aufgerufen werden. %TODO: Referenziere Section zu Getter und Setter

Um die Objekte beim Sortieren durch den Mergesort miteinander vergleichen zu können, werden die Methoden des Trait \texttt{Ordered} implementiert. Dieser Trait entspricht dem Interface \texttt{Comparable} in Java. Zusätzlich zu der Methode \texttt{compare} werden verschiedene Vergleichsoperatoren für das Objekt SortElement neu definiert:

\clearpage

\chapter{Zusammenfassung \& Fazit}\label{chap:zusammenfassung-und-fazit}
\section{Reflexion des Vorgehens}\label{sec:reflexion}
\section{Kritische Betrachtung}\label{sec:kritische-betrachtung}
\section{Fazit}\label{sec:fazit}

\clearpage

\addcontentsline{toc}{chapter}{Literatur}
\bibliography{sources}
\bibliographystyle{alpha}
\end{document}
