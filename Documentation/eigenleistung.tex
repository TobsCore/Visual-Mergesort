\chapter{Eigenleistung}\label{chap:eigenleistung}

\section{Das Objekt VisualMergesort}
Das Objekt VisualMergesort ist ein Singleton-Objekt der anonymen Klasse VisualMergesort, welches garantiert einzigartig ist. Das bedeutet, dass keine weiteren Objekte gleicher Art erzeugt werden können. „VisualMergesort“ erbt von JFXApp und definiert somit den Startpunkt unserer Applikation.

\begin{lstlisting}[language=Scala]
import java.io.IOException

import scalafx.application.JFXApp
import scalafx.application.JFXApp.PrimaryStage
import scalafx.scene.Scene
import scalafx.Includes._
import scalafxml.core.{FXMLView, NoDependencyResolver}
object VisualMergesort extends JFXApp {

  private val layoutFile: String = "/VisualMergesort.fxml"
  val resource = getClass.getResource(layoutFile)

  if (resource == null) {
    throw new IOException(s"Cannot load resource: $layoutFile")
  }

  val root = FXMLView(resource, NoDependencyResolver)

  stage = new PrimaryStage() {
    title = "Visual Mergesort"
    scene = new Scene(root)
  }

}
\end{lstlisting}

In diesem Objekt wird das in FXML vordefinierte Layout unserer GUI in die Stage der Applikation geladen.

\section{Die Klasse SortElement}
Die Klasse SortElement ist für die Objekte verantwortlich, die durch den Algorithmus auf der Zeichenfläche sortiert werden sollen. Die zu sortierenden Objekte sind zusammengesetzte Gruppen und bestehen aus einem Rechteck sowie einem darunter liegenden Text, welcher den Wert des Objektes angibt. Die Höhe des Rechtecks verhält sich proportional zu diesem Wert.
Der Text benötigt, um mittig unter dem Rechteck zu stehen, noch einen offset, falls der Wert einstellig ist.

%TODO: Caption
\begin{lstlisting}[language=Scala]
class SortElement(val number: Int, var _xPos: Double, var _yPos: Double) extends Group with Ordered[SortElement]  {
  require(number >= 1 && number <= 99 , "the number must be between 1 and 99 (inclusive)")

  var text = new Text(number.toString)
  text.style = "-fx-font-size: 10px; -fx-background: #f00"

  val offset = if (number < 10) { SortElement.smallNumberOffset } else { 0 }
  text.translateX() = xPos + offset
  text.translateY() = _yPos + number + SortElement.width

  var rectangle = new Rectangle(new javafx.scene.shape.Rectangle(_xPos, _yPos, SortElement.width, number))
  rectangle.setFill(Color.DARKBLUE)
  this.getChildren.addAll(rectangle,text)
\end{lstlisting}

Um die von Scala vordefinierten Getter und Setter der Variablen \texttt{xPos} und \texttt{yPos} zu überschreiben, werden diese in \texttt{\_xPos} und
\texttt{\_yPos} umbenannt. Danach werden zunächst die Getter definiert:

\begin{lstlisting}[language=Scala,caption=Definiert die Setter Methoden]
def xPos = _xPos
def yPos = _yPos
\end{lstlisting}

Dieser Code definiert zwei simple Methoden \texttt{xPos} und \texttt{yPos}, welche die Variablen \texttt{\_xPos} und \texttt{\_yPos} zurückgeben. Da bei Scala der letzte Ausdruck einer Methode auch gleichzeitig der Rückgabewert ist und geschweifte Klammern nicht benötigt werden, falls die Methode nur aus einem Ausdruck besteht, sind diese und das Return-Statement nicht vorhanden.

Als nächstes werden die Setter neu definiert:

%TODO: Caption
\begin{lstlisting}[language=Scala]
def xPos_= (x: Double) {
  _xPos = x
  text.translateX = _xPos + offset
  rectangle.x = x
}

def yPos_= (y: Double) {
  _yPos = y
 text.translateY = _yPos + number + SortElement.width
  rectangle.y = y
}
\end{lstlisting}

Der Name dieser Methoden ist \texttt{xPos\_} und \texttt{yPos\_}. Der Unterstrich dient in Scala als spezielles Zeichen und ist in diesem Fall als Platzhalter zu verstehen, der beim Aufrufen der Methoden durch ein Leerzeichen ersetzt werden kann. Somit können diese Methoden im folgenden Code sowohl mit \texttt{xPos =}  bzw. \texttt{yPos =} als auch mit \texttt{xPos\_=} bzw. \texttt{yPos\_=} aufgerufen werden. %TODO: Referenziere Section zu Getter und Setter

Um die Objekte beim Sortieren durch den Mergesort miteinander vergleichen zu können, werden die Methoden des Trait \texttt{Ordered} implementiert. Dieser Trait entspricht dem Interface \texttt{Comparable} in Java. Zusätzlich zu der Methode \texttt{compare} werden verschiedene Vergleichsoperatoren für das Objekt SortElement neu definiert:
